\chapter{Struktur und Inhalt}
\label{chap:struktur-und-inhalt}


\section{Wissenschaftliche Arbeiten im technischen Bereich}

Die meisten Arbeiten in den technischen Studiengängen der FHDW befassen sich mit technischen Fragestellungen aus dem Bereich Hardware oder Softwaretechnik. Jedoch sind auch mathematische oder andere Schwerpunkte möglich. In diesem Kapitel beschreiben wir, welche Zielsetzung solche Arbeiten im Allgemeinen haben und welche Konsequenzen sich deshalb für die Strukturierung des Inhalts der Arbeiten ergeben.

Dabei ist die eine Unterscheidung der Prüfungsformen wichtig, da für Praxisarbeiten und Lehrprojekte andere Anforderungen gelten als für die Bachelor- und Master-Thesis.

\section{Praxisarbeit und Lehrprojekt}

\subsection{Zielsetzung}

Die Zielsetzung von Praxisarbeiten, Lehrprojekten (Master) und Transferprojekten (Aufbaustudium) ist typischerweise die Anwendung von Fachwissen aus dem Studium für die Lösung einer praktischen Problemstellung.

Bei Praxisarbeiten hat \enquote{...die Entwicklung eines Werkstücks mittlerer Komplexität
(Hardware- und / oder Software-Teilsystem) zum Gegenstand.} \cite[S.\,122]{FHDWModulhabdbuchBachelor}. Für Lehrprojekte (Master) ist die Zielsetzung ähnlich: \enquote{Im Lehrprojekt sollen die in der Theorie erlernten Konzepte und Methoden in einem konkreten Geschäftsfeld praktisch angewendet werden}~\cite[S.\,12]{FHDWModulhabdbuchMasterInformationEngineering}.

\subsection{Struktur}
\label{sec:struktur-projektbericht}

Aufgrund der Zielsetzung sind die Arbeiten typischerweise so aufgebaut, dass zunächst eine bestimmte Problemstellung analysiert wird, dann ein Lösungskonzept erarbeitet wird und letztlich die praktische Umsetzung (Implementierung) dieser Lösung beschrieben wird, möglichst mit einer anschließenden Evaluierung -- der Beurteilung der Qualität der Lösung.

Somit ist die kanonische Struktur einer Praxisarbeit oder eines Lehrprojektberichts folgende:

\begin{enumerate}
    \item \textbf{Einleitung}: 
    
    \begin{enumerate}
        \item \textbf{Problemstellung}:  Was ist der Problemkontext? Was ist das konkrete Problem? Was sind spezifische Anforderungen an die Lösung des Problems (wie vielleicht Anforderungen an Software, \url{https://de.wikipedia.org/wiki/ISO/IEC_9126})?
        \item \textbf{Lösungsansatz}: Wie wird das Problem innerhalb der Arbeit gelöst? Inwiefern löst der Lösungsansatz bestimmte Aspekte des Problems oder erfüllt spezifische Anforderungen? Was sind ggf. besonders hervorzuhebende Eigenschaften der Lösung?
        \item \textbf{Ergebnis der Arbeit}: Was sind die wichtigste Ergebnisse der Arbeit und warum sind diese Ergebnisse gut?
        \item \textbf{Struktur der Arbeit}: Wie ist die Arbeit aufgebaut?
    \end{enumerate}
    \item \label{struktur:grundlagen} \textbf{Grundlagen}: Hier erklären Sie Konzepte, Verfahren und Technologien, die für das Verständnis der folgenden Kapitel notwendig sind. Wissen, dass sie selbst im Studium schon erlernt haben, können Sie als gegebenes Wissen voraussetzen (siehe Bewertungskriterium \textit{Richtiges Niveau für die Zielgruppe} in Abschnitt~\ref{sec:inhaltliche-kriterien}).
    \item \label{struktur:problemanalyse} \textbf{Problemanalyse}: Hier erarbeiten Sie Anforderungen an eine Lösung und legen auch dar, warum es bisher für dieses Problem noch keine Lösung gibt oder existierende Lösungen unzureichend sind (Abgrenzung zu verwandten Lösungen/Arbeiten).
    \item \label{struktur:loesungskonzept} \textbf{Lösungskonzept}: Hier beschreiben Sie die Lösung des Problems auf konzeptueller Ebene, möglichst unabhängig von der verwendeten Technologie oder Programmierung. Gegebenenfalls wägen Sie Vor- und Nachteile verschiedener Lösungsalternativen ab. Hier werden Architekturen, Abläufe oder Algorithmen beschrieben, wünschenswerterweise mit entsprechenden fachlichen Diagrammen (zum Beispiel UML).
    \item \label{struktur:umsetzung-und-evaluierung} \textbf{Umsetzung und Evaluierung}: Während der vorige Abschnitt die Lösung auf konzeptueller Ebene beschreiben sollte, beschreiben Sie hier wichtige technologische Aspekte, die nicht zum konzeptuellen Teil der Lösung gehören. Zum Beispiel welche Frameworks, Plattformen, Programmiersprachen oder Bibliotheken haben Sie verwendet? Sie können hier auch Teile von Programmcode beschreiben -- aber nur wenn programmiertechnische Details im Vergleich zur konzeptuellen Beschreibung der Lösung einen wichtigen Beitrag darstellen. Ebenso sollten Sie in diesem Kapitel bewerten, inwiefern Ihre Lösung die vorher definierten Anforderungen erfüllt (siehe Problemstellung). Sind zum Beispiel die Performance, Fehlertoleranz, Anpassbarkeit oder Benutzerfreundlichkeit wichtige Anforderungen, können Sie hier Ergebnisse von entsprechenden Analysen darlegen.
    \item \label{struktur:zusammenfassung-ausblick} \textbf{Zusammenfassung und Ausblick}: Hier fassen Sie die wichtigsten Beiträge Ihrer Arbeit nochmals zusammen: Was sind die wichtigsten Ergebnisse? Warum sind diese Ergebnisse gut? Warum ist die Arbeit ein Erfolg? Zudem können Sie beschreiben, wie Ihre Lösung in Zukunft noch verbessert oder weiterentwickelt werden kann (Ausblick). 
    \item \textbf{Kritische Reflexion (\textit{Optional})}: Hier können Sie beschreiben, wie Sie bestimmte Herausforderungen gemeistert haben, was Sie beim nächsten Mal anders machen würden oder welche Empfehlungen Sie Dritten geben möchten, die eine ähnliche Problemstellung bearbeiten.
\end{enumerate}


\section{Bachelor- und Master-Thesis}

\subsection{Zielsetzung}

Die Anforderungen an eine Bachelor- oder Master-Thesis sind andere als die Anforderungen an Praxisarbeiten oder Lehrprojektberichte.

Bei einer Bachelor- oder Master-Thesis sollen Sie Fachwissen aus dem Studiengang angewenden, um eine komplexe Problemstellung aus dem Themengebiet des jeweiligen Studiengangs zu bearbeiten, \cite{FHDWModulhabdbuchBachelor,FHDWModulhabdbuchMasterInformationEngineering}. 
Das Ziel dabei ist die \textbf{Beantwortung einer wissenschaftlichen Fragestellung} und somit die Erlangung einer \textbf{wissenschaftlichen Erkenntnis}.

Sie sollen also demonstrieren, dass Sie in der Lage sind, für ein Problemgebiet Forschungsfragen zu definieren, geeignete Forschungsmethoden auszuwählen und anzuwenden und letztlich Erkenntnisse abzuleiten.


Doch was ist eine wissenschaftlichen Erkenntnis in den technischen Wissenschaften? Die technischen Wissenschaften beschäftigen sich mit \textbf{\textit{Methoden zur Entwicklung eines technischen Produkts oder Prozesses, die die Qualität des Produkts oder Prozesses erhöht oder die Effizienz der Entwicklung steigert.} }

Neue Erkenntnisse über Methoden speziell in der Informatik in diesem Sinne sind zum Beispiel:

\begin{itemize}
    \item Erkenntnisse über den Einsatz von Algorithmen und Technologien \textit{in} einer zu entwickelnden Hardware oder Software.
    \item Erkenntnisse über den Einsatz von Algorithmen, Technologien, Werkzeugen \textit{bei der Entwicklung} von Hardware oder Software.
    \item Erkenntnisse über Entwicklungsprozesse oder Organsisationsaspekte im Kontext der Hard-/Softwareentwicklung (menschliche oder sozio-technische Aspekte).
\end{itemize}


Darüber Hinaus sind wissenschaftliche Erkenntnisse in technischen Wissenschaften auch:

\begin{itemize}
    \item Erkenntnisse über theoretische Grundlagen im technischen Fachgebiet, z.B. mathematische Beweise.
    \item Erkenntnisse an interdisziplinären Schnittstellen zu anderen Themengebieten, wie zum Beispiel Naturwissenschaften, Ethik, oder Wirtschaftswissenschaften.
    \item Eine Erkenntnis über den wissenschaftlichen Erkenntnisgewinn selbst (metawissenschaftlicher Beitrag).
\end{itemize}



\subsubsection{Wissenschaftlich Arbeit oder Projektbericht?}

Ein wissenschaftlicher Beitrag kann eine allgemeine oder spezielle Erkenntnis über eine entwicklungsmethodische Problemstellung sein. Es kann eine theoretische Erkenntnis sein, die vielleicht erst indirekt zur Verbesserung einer entwicklungsmethodische Problemstellung beiträgt. Aber es kann auch eine Erkenntnis über Methoden für das Lösen eines speziellen Problems im Unternehmenskontext sein.

Aber wo ist genau der Unterschied zwischen einem Projektbericht und einer Bachelor- oder Master-Thesis?
Während ein Projektbericht eher die Entwicklung eines einzelnen Werkstücks beschreibt, sollte eine Bachelor-Thesis und eine Master-Thesis eine allgemeinere, methodische Fragestellung betrachten. 

\textbf{Beispiel}: Nehmen wir an, Sie haben eine Projektarbeit durchgeführt mit dem Titel \enquote{Entwicklung eines Adapters für die Anbindung eines Legacy-Systems für die Kundendatenverwaltung}. Im Rahmen der Projektarbeit haben Sie die Anforderungen an einen spezifische Adapter-Softwarekomponente analysiert und dann eine passende Lösung entwickelt. Bearbeiten Sie eine ähnliche Fragestellung in einer Bachelor-Thesis, würden Sie hingegen eher eine allgemeinere Fragestellung bezüglich der Entwicklungsmethodik für solche Adapter-Softwarekomponenten behandeln. Sie würden sich zum Beispiel fragen, welche Entwurfsmuster  allgemein bei der Entwicklung solcher Adapter-Softwarekomponenten angewendet werden sollten oder welche Herangehensweisen es bei der Entwicklung solcher Komponenten vorteilhaft sind (Wie erhebe ich Anforderungen? Wie teste ich? ...). Diese Fragen können und sollen Sie gerne mit Hilfe einer konkreten Instanz des Problems (zum Beispiel eines konkreten Adapters für die Kundendaten) beantworten. Die Frage sollte jedoch sein: \textit{Welche Erkenntnisse aus der Entwicklung der einen Lösung lassen sich auf die Entwicklungen von ähnlichen Lösungen in der Zukunft übertragen?} Daher würde der Titel einer entsprechenden Bachelor-Thesis gegebenenfalls eher lauten \enquote{Entwurfsmuster für Entwicklung von Adapter-Komponenten für die Anbindung von Legacy-Systemen} oder \enquote{Qualitätssicherungsmethoden bei der Entwicklung von Adapter-Komponenten für die Anbindung von Legacy-Systemen}.

\subsection{Erlangung wissenschaftlicher Erkenntnisse: konstruktiv oder analytisch}

Eine wissenschaftliche Erkenntnis kann \textit{konstuktiv} oder \textit{analytisch} erlangt werden. 
Ein konstruktiver Erkenntnisgewinn entsteht durch die Entwicklung und Bewertung einer neuartigen Lösung für eine entwicklungsmethodische Problemstellung.
Ein analytischer Erkenntnisgewinn entsteht durch Untersuchung von existierenden Aspekten und Zusammenhängen in einem Problembereich, zum Beispiel existierender Technologien oder Entwicklungsmethoden.

Das systematische Vorgehen beim \textbf{konstruktiven} Erkenntnisgewinn ist folgendes:
\begin{itemize}
    \item Eine Problem in einer Hard-/Softwareentwicklungsmethode identifizieren.
    \item Darlegen, inwiefern dieses Problem die Effizienz der Entwicklung behindert oder die Qualität von Hard-/Software beeinträchtigt.
    \item Eine Lösung vorschlagen und begründen, warum diese Lösung eine gute und neuartige Lösung des Problems darstellt, zum Beispiel mithilfe einer logischen oder experimentellen Beweisführung.
\end{itemize}

Das systematische Vorgehen beim \textbf{analytischen} Erkenntnisgewinn ist folgendes:
\begin{itemize}
    \item Eine Wissenslücke bezüglich einer Hard-/Softwareentwicklungsmethode identifizieren.
    \item Darlegen, inwiefern das Schließen dieser Wissenslücke zur Verbesserung von Hard-/Softwareentwicklungsmethoden beitragen kann.
    \item Erkenntnisse erlangen, die die identifizierte Wissenslücke schließen (zum Beispiel über Experimente, Umfragen, Literaturrecherche, etc.) und die Verlässlichkeit der Ergebnisse begründen.
\end{itemize}



\subsection{Struktur (konstruktive Arbeit)}

Ein Großteil der Abschlussarbeiten im technischen Bereich verfolgt einen konstruktiven Erkenntnisgewinn. Eine solche Arbeit hat typischerweise folgende Struktur:

\begin{enumerate}
    \item \textbf{Einleitung}
    \begin{enumerate}
        \item \textbf{Problemstellung}: Was ist der Problemkontext? Was ist das (methodische) Problem? Was sind spezifische Anforderungen an die Lösung des Problems?
        \item \textbf{Abgrenzung zu verwandten Arbeiten}: Inwieweit sind bestimmte Aspekte der Problemstellung bereits gelöst und welche noch nicht? Gehen Sie kurz auf entsprechende Quellen ein.
        \item \textbf{Forschungsfrage}: Welche Forschungsfrage ergibt sich aus der Problemstellung und soll in dieser Arbeit beantwortet werden?
        \item \textbf{Forschungsmethode}: Hier erklären Sie, dass Sie eine \textit{konstruktive} Forschungsmethode anwenden. Beschreiben Sie den \textbf{Lösungsansatz} für die Problemstellung, den Sie vorschlagen und welche \textbf{Evaluierungssmethode} Sie für die Bewertung des Lösungsansatzes anwenden, zum Beispiel Evaluierung mittels einer prototypischen Umsetzung.
        \item \textbf{Ergebnis und Beitrag der Arbeit}: Was sind die wichtigste Erkenntnisse der Arbeit und warum sind diese Ergebnisse gut?
    \end{enumerate}
    \item \textbf{Grundlagen}: siehe Beschreibung zu \hyperref[struktur:grundlagen]{\textit{Grundlagen}}  in Abschnitt~\ref{sec:struktur-projektbericht}
    \item \textbf{Problemanalyse}: siehe Beschreibung zu \hyperref[struktur:problemanalyse]{\textit{Problemanalyse}}  in Abschnitt~\ref{sec:struktur-projektbericht}
    \item \textbf{Lösungskonzept}: siehe Beschreibung zu \hyperref[struktur:loesungskonzept]{\textit{Lösungskonzept}}  in Abschnitt~\ref{sec:struktur-projektbericht}
    \item \textbf{Umsetzung und Evaluierung}: siehe Beschreibung zu \hyperref[struktur:umsetzung-und-evaluierung]{\textit{Umsetzung und Evaluierung}}  in Abschnitt~\ref{sec:struktur-projektbericht}
    \item \textbf{Verwandte Arbeiten}: Hier grenzen Sie Ihre Arbeit zu vergleichbaren existierenden Arbeiten ab, siehe Abschnitt~\ref{sec:verwandte-arbeiten}.
    \item \textbf{Zusammenfassung und Ausblick}:
    siehe Beschreibung zu \hyperref[struktur:zusammenfassung-ausblick]{\textit{Zusammenfassung und Ausblick}}  in Abschnitt~\ref{sec:struktur-projektbericht}
\end{enumerate}


\subsection{Struktur (analytische Arbeit)}

Eine analytische Arbeit hat typischerweise folgende Struktur:

\begin{enumerate}
    \item \textbf{Einleitung}
    \begin{enumerate}
        \item \textbf{Problemstellung} (s.o.)
        \item \textbf{Abgrenzung zu verwandten Arbeiten}: (s.o.)
        \item \textbf{Forschungsfrage}: (s.o.)
        \item \textbf{Forschungsmethode}: Hier erklären Sie, dass Sie eine \textit{analytische} Forschungsmethode anwenden. Beschreiben Sie welche \textbf{Forschungsmethode} Sie für die Beantwortung der Forschungsfragen anwenden (zum Beispiel Experiment, Interviews).
        \item \textbf{Ergebnis und Beitrag der Arbeit}: Was sind die wichtigste Erkenntnisse der Arbeit und warum sind diese Ergebnisse gut?
    \end{enumerate}
    
    \item \textbf{Grundlagen}: siehe Beschreibung zu \hyperref[struktur:grundlagen]{\textit{Grundlagen}}  in Abschnitt~\ref{sec:struktur-projektbericht}
    \item \textbf{Problemanalyse}: siehe Beschreibung zu \hyperref[struktur:problemanalyse]{\textit{Problemanalyse}}  in Abschnitt~\ref{sec:struktur-projektbericht}
    \item \textbf{Forschungsmethode}: Hier beschreiben Sie im Detail, wie Sie die Analyse durchgeführt haben.
    \item \textbf{Ergebnisse}: Hier präsentieren Sie die Ergebnisse Ihrer Arbeit und beschreiben inwiefern diese Ergebnisse verlässlich sind und welche Faktoren die Gültigkeit der Ergebnisse gegebenenfalls beeinflussen.
    \item \textbf{Verwandte Arbeiten}: Hier grenzen Sie Ihre Arbeit zu vergleichbaren existierenden Arbeiten ab, siehe Abschnitt~\ref{sec:verwandte-arbeiten}.
    \item \textbf{Zusammenfassung und Ausblick}:
    siehe Beschreibung zu \hyperref[struktur:zusammenfassung-ausblick]{\textit{Zusammenfassung und Ausblick}}  in Abschnitt~\ref{sec:struktur-projektbericht}    
    
\end{enumerate}

\section{Was ist eine Forschungsfrage?}

Eine Forschungsfrage beschreibt, welche Frage innerhalb der Arbeit beantwortet werden soll. Sie beschreibt die Zielsetzung der Arbeit. 

Die Forschungsfrage sollte beim Beginn der Arbeit feststehen und mit dem Prüfer der Arbeit besprochen werden.

Die Forschungsfrage sollte möglichst präzise formuliert sein. Ein Beispiel für eine \textbf{konstruktive} Forschungsfrage ist \enquote{Wie können effizienter Lasttests für die Bewertung der Performance einer Software erstellt werden?} Hier ist allerdings einiges unklar. Was für eine Software? Was ist der Problembereich? Effizienter als was? Eine präzisere und bessere Forschungsfrage wäre \enquote{Wie können Lasttests für die Bewertung der Performance von datenintensiven REST-Services auf Basis von aufgezeichneten Lastprofilen in Log-Dateien automatisiert erstellt werden.}

Ein Beispiel für eine \textbf{analytische} Forschungsfrage ist \enquote{Steigert ein agiles Vorgehen die Effizienz der Softwarentwicklung?} Hier ist wieder einiges unklar: Welche agile Vorgehen genau? Entwicklung von was? Welche Rahmenbedingungen sind gegebenenfalls wichtig? Eine Präzisierung der Forschungsfrage wäre zum Beispiel \enquote{Steigert ein Vorgehen nach SCRUM die Effizienz der Entwicklung von Microservices durch crossfunktionale Entwicklerteams im Versicherungsbereich?}

Eine Forschungsfrage sollte möglichst systematisch nachvollziehbar beantwortbar sein. Dazu kann eine Forschungsfrage in Unterfragen unterteilt werden, die konkret beurteilbar oder messbare sind, zum Beispiel: \enquote{Wie lange dauert das automatisierte Erstellen von Lasttests im Vergleich zur manuellen Entwicklung der Lasttests?} oder \enquote{Im Vergleich zu historischen Daten über nicht-agile Entwicklungsprojekte, steigert ein Vorgehen nach SCRUM die Zahl der Funktionen, die in einem festgelegten Zeitraum entwickelt werden?}

Die Unterteilung der Forschungsfrage in Unterfragen sollte möglichst mit der Evaluierung korrelieren, die Sie im Rahmen Ihrer Arbeit durchführen. Dabei kann die genaue Formulierung der Unterforschungsfragen sich durchaus erst im Laufe der Bearbeitungszeit ergeben und davon abhängen, welche Evaluierungen Sie letztlich im Rahmen Ihrer Arbeit durchführen.

Es kann mehrere Forschungsfragen geben, die miteinander in Verbindung stehen, zum Beispiel \enquote{Steigert ein Vorgehen nach SCRUM die Entwicklerzufriedenheit}. Es sollten jedoch nicht mehr als zwei oder drei Hauptforschungsfragen sein. Inhaltliche Tiefe ist meist besser als Breite.


\section{Was soll in den Grundlagen stehen?}

Oft stellt sich die Frage: Was soll alles in den Grundlagen stehen? Die Antwort darauf ist, dass Sie alle Konzepte, Verfahren und Technologien erklären sollten, die für das Verständnis der folgenden Kapitel notwendig sind und nicht bereits bei Studierenden in Ihrem Studienabschnitt als  bekannt vorausgesetzt werden können.

Die Zielgruppe Ihrer Arbeit ist also nicht der Prüfer oder Betreuer, sondern andere Studierende in Ihrem Studienabschnitt. Werden Sie sich dieser Zielgruppe bewusst und finden Sie in der Arbeit das richtige Niveau für Ihre Zielgruppe.

\section{Verwandte Arbeiten?}
\label{sec:verwandte-arbeiten}

Forschungsarbeiten stehen immer im Bezug zu vorangegangener Forschung. Im Kapitel Verwandte Arbeiten sollten Sie Ihre Arbeit zu Quellen abgrenzen, die im Bezug zu Ihrer Arbeit stehen. Das können Arbeiten sein, die eine ähnliche Problemstellung oder Lösungsansatz behandeln. Wichtig ist, dass Sie die verwandten Arbeiten nicht nur nennen, sondern auch erklären worin die Unterschiede zu Ihrer Arbeit liegen.

In einer Bachelor-Thesis sollte das Kapitel Verwandte Arbeiten eine halbe bis ganze Seite lang sein, in einer Master-Thesis mindestens eine Seite.

Wie viele verwandte Arbeiten Sie nennen, hängt vom Thema ab. Es sollten mindestens drei wissenschaftliche Quellen sein. Wenn Sie keine verwandten Arbeiten finden, kann das ein Hinweis darauf sein, dass Ihr Thema gegebenenfalls keine gute wissenschaftliche Fragestellung im o.g. Sinne ist. Sprechen Sie darüber mit Ihrem Erstprüfer.

\section{Inhaltsverzeichnis, Abkürzungsverzeichnis und Glossar}
\label{sec:inhaltsverzeichnis-und-glossar}

Die Projektarbeit oder Thesis sollte in jedem Fall ein Inhaltsverzeichnis aufweisen (wie zum Beispiel in dieser Vorlage). 

Verwenden Sie in der Arbeit viele fach- oder firmenspezifische Begriffe oder Abkürzungen, so sollten diese in einem Glossar und/oder Abkürzungsverzeichnis erläutert werden. Wenn Sie ein Glossar und/oder ein Abkürzungsverzeichnis verwenden, sollten Sie dennoch die Bedeutung eines Fachbegriffs oder einer Abkürzung bei dessen erstem Auftreten im Text erläutern.  Gängige und bekannte Fachbegriffe oder Abkürzungen, wie zum Beispiel \textit{Testfall} oder \textit{GmbH}, brauchen und sollten Sie  nicht in einem Glossar und/oder Abkürzungsverzeichnis aufnehmen.