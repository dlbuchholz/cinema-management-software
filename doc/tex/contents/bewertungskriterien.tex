\chapter{Bewertungskriterien}
\label{chap:bewertungskriterien}

Auf folgende Punkte achten wir unter anderem bei der Bewertung (die Reihenfolge drückt keine Gewichtung aus):

\section{Inhaltliche Kriterien}
\label{sec:inhaltliche-kriterien}

    \begin{itemize}
    
        \item \textbf{Methodische Fundierung}: Wird das in der Arbeit adressierte Problem klar beschrieben (Motivation)? Wird eine geeignete und passende Forschungsfrage formuliert und nachvollziehbar beantwortet? Wird klar,warum der/die Autor/-in den verfolgten Forschungs-/Lösungsansatz gewählt hat?
        
        \item \textbf{Fachliche Fundierung}: Wird Fachwissen aus dem Studium (oder darüber hinaus) geeignet angewendet? z.B. Algorithmen, Entwurfsmethoden, Technologien, passende fachliche Diagramme für die Beschreibung von Architekturen und Abläufen?
        
	    \item \textbf{Abgrenzung zu verwandten Arbeiten}: Werden verwandte Arbeiten genannt und auch mit der eigenen Arbeit verglichen: Was ist in der eigenen Arbeit anders, neu oder besser als in verwandten Arbeiten?
	    
	    \item \textbf{Erweitertes Problemverständnis}: Demonstriert der/die Autor/-in ein umfassendes Verständnis des Problembereichs? Sind die genannten Probleme in einen größeren Zusammenhang gesetzt? Werden Verknüpfungen zu anderen Themen hergestellt? Wird ein interessanter Ausblick gegeben? (\enquote{über den Tellerrand geschaut})
        
        \item \textbf{Struktur}: Ist die Arbeit insgesamt gut strukturiert? Sind die einzelnen Abschnitte, insbesondere die Einleitung, gut strukturiert? Sind Argumentationen und Erklärungen von Problemen, Konzepten und Beispielen logisch strukturiert?
	
		\item \textbf{Verständlichkeit, Qualität von Beispielen}: Ist der Text verständlich geschrieben? Ist ein Beispiel vorhanden, das das Problem und den Lösungsansatz passend beschreibt? Es wirkt sich positiv aus, wenn das Beispiel selbst erdacht ist.
		
	    \item \textbf{Richtiges Niveau für die Zielgruppe}: Ist das Niveau richtig für die Zielgruppe? Ist der Problembereich gut beschrieben? Sind alle nötigen Grundlagen erklärt? Es kann sich negativ auswirken, wenn Wissen über Konzepte vorausgesetzt wird, die Studierende im selben Studienabschnitt noch nicht gelernt haben.

        \item \textbf{Qualität des Ergebnisses, Erkenntnisgewinn}: Wie hoch ist die Qualität der Lösung? Erfüllt die Lösung vorher definierte Anforderungen? Insbesondere bei Bachelor-/Master-Theses: Wie substantiell ist der Erkenntnisgewinn im Sinne der gestellten Forschungsfrage(n)?
        
        \item \textbf{Eigener kreativer Anteil}: 
        Hat der/die Autor/-in eigene Ideen in die Arbeit eingebracht oder besondere Leistungen gezeigt, die zu einem deutlichen Erkenntnisgewinn beigetragen haben (z.B. eigene Beispiele überlegt, Fallstudien oder Experimente entworfen)? Geht das Engagement über die vereinbarten Minimalanforderungen hinaus?
        
    \end{itemize}
	
\section{Äußerliche und formale Kriterien}

    \begin{itemize}

	    \item \textbf{Schreibstil, Sprachlicher Ausdruck}: Werden Fachbegriffe präzise definiert und verwendet? Gute Wortwahl und Betonung von wichtigen Punkten werden positiv bewertet. Ungenaue Erklärungen, redundante Erklärungen, Rechtschreib- und Grammatikfehler können sich negativ auswirken.
	    
    	\item \textbf{Gestaltung}: Macht die Arbeit insgesamt einen ansprechenden und einheitlichen Eindruck? Hat sie ein gut lesbares Schriftbild? Enthält die Arbeit lesbare und ansprechende Grafiken? Werden Auflistungen und Tabellen verwendet, wo diese Sinn machen?
	
    	\item \textbf{Arbeit mit Quellen}: Werden Quellen an passenden Stellen genannt und korrekt zitiert?
    \end{itemize}
    
\section{Bewertungskriterien für das Kolloquium}
\label{sec:bewertung-kolloquium}

Bei Praxis- und Abschlussarbeiten (mit Ausnahme des Lehrprojekts im Master) wird zusätzlich zur Ausarbeitung ein Kolloquium abgehalten und bewertet. Besprechen Sie mit Ihrem (Erst-)prüfer, in welcher Form er/sie das Kolloqium abhalten möchte. Bei Praxisprojekten ist es üblich, einen Vortrag mit Vortragsfolien zu halten, mit anschließenden kurzer Diskussion. Bei dem Kolloquium zur Bachelor- oder Master-Thesis kann es auch einen Vortrag inklusive Diskussion geben oder ein \textit{Prüfungsgespräch}, in dem Ihnen die Prüfer Fragen zur Arbeit stellen. In einem Prüfungsgespräch wird die fachliche und methodische Fundierung Ihrer Antworten bewertet und ob Sie Zusammenhänge gut verständlich erklären können.
Auf folgende Kriterien achten wir bei der Bewertung des Vortrags:

    \begin{itemize}

	    \item \textbf{Struktur}: Ist der Vortrag gut strukturiert? Wird insbesondere die Problemstellung klar beschrieben und der Lösungsansatz und die Ergebnisse nachvollziehbar erklärt?
	    
	    \item \textbf{Verständlichkeit}: Werden Fachbegriffe und nötige Grundlagen erläutert? Ist das Niveau richtig für die Zielgruppe? Wird ein geeignetes Beispiel verwendet? Sind Abbildungen verständlich gestaltet?
	    
    	\item \textbf{Vortragsstil und Foliengestaltung}: Werden die Inhalte verständlich vorgetragen? Sind die Vortragsfolien ansprechend gestaltet? Sind Texte und Abbildungen gut lesbar?
    	
    	\item \textbf{Fachliche und Methodische Fundierung}: Wird fachliches und methodisches Wissen demonstriert? Werden zum Beispiel  passende fachlicher Diagrammen für die Beschreibung von Architekturen und Abläufen verwendet und erklärt? Antwortet der/die Vortragende fundiert auf Fragen?
	
    \end{itemize}
    