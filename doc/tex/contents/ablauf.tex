\chapter{Ablauf}

Generell gilt:
\begin{itemize}
    \item Themen von Praxis- und Abschlussarbeiten sind mit dem jeweiligen Betreuer abzusprechen und im FHDW-Portal unter \url{https://my.fhdw-hannover.de/} anzumelden.
    \item Arbeiten müssen vor dem Ablauf der Abgabefrist digital im Portal \url{https://my.fhdw-hannover.de/} unter \textit{Abgaben} hochgeladen werden. \textbf{Zur Abgabe gehören folgende Dokumente}. Diese sind als separate Dateien hochzuladen:
    \begin{enumerate}
        \item Das Hauptdokument der Arbeit als PDF, inklusive der Anhangs-Kapitel (Appendizes), sofern vorhanden.
        \item Die \textit{Ehrenwörtliche Erklärung}, eigenhändig  unterschrieben und eingescannt als PDF.
        \item Bei Bedarf weitere Dokumente wie Fragebögen, Rohdaten, Source\-code oder ähnliches als ZIP-Datei.
    \end{enumerate}
    \item In allen Fällen von Terminüberschreitungen wird die Arbeit mit \textit{nicht ausreichend} bewertet!
    \item Eine Praxisarbeit kann bei Nichtbestehen jeweils einmal wiederholt werden. Darüber hinaus gelten im Einzelnen die unten aufgeführten Regeln.
\end{itemize}

\section{Praxisprojekt}
\label{sec:praxisprojekt}

\begin{itemize}
    \item Zwei obligatorische Praxisprojekte sind im 2., 3., 4. oder 5. Praxisquartal zu erstellen. Empfohlen wird die Anfertigung im 3. und 4. Praxisquartal.
    \item Die reine Bearbeitungszeit der Praxisprojekte beträgt \textbf{6 Wochen}, die sich beliebig innerhalb der 12 Wochen des Praxisquartals verteilen können. (Es werden 9 CP vergeben, das entspricht einer Arbeitszeit von $9*25=225$ Stunden, also etwa 6 Woche Arbeitszeit.)
    \item Das Thema kann in Ihrem Partnerunternehmen vergeben werden und ist rechtzeitig vor Bearbeitungsbeginn mit einem von Ihnen gewählten Prüfer abzusprechen. Erster Ansprechpartner als Prüfer ist ihr Mentor.
    \item Den Bearbeitungsbeginn und alle weiteren Termine entnehmen Sie dem jeweiligen Zeitplan. Eine Verlängerung des Abgabetermins ist nicht möglich. Im Falle von Terminüberschreitungen wird die Arbeit mit \textit{nicht ausreichend} bewertet.
    \item Mit dem FHDW-Betreuer werden im Allgemeinen zwei Termine während der Anfertigungszeit durchgeführt: Der erste Termin dient dazu, den geplanten Inhalt der Arbeit anhand eines Inhaltsverzeichnisses zu diskutieren und so zu sichern, dass die Arbeit fachgerecht aufgebaut ist. Vor dem zweiten Termin soll der Betreuer die Gelegenheit für eine Leseprobe erhalten. Dazu sollen zwischen 1/3 und 2/3 der Arbeit fertiggestellt sein. Im Termin selbst können die bisherigen Inhalte dann diskutiert werden. Die Termine finden im Allgemeinen in der FHDW statt, können aber auch online oder im Partnerunternehmen, ggf. mit Beteiligung weiterer Personen im Unternehmen, stattfinden.
    \item Das Kolloquium findet in der Regel im folgenden Theoriequartal statt. Es soll die Praxisarbeit inhaltlich in einem 20-minütigen Vortrag. mit 10 Minuten anschließender Diskussion vor Publikum präsentieren. Ein erfolgreich erbrachtes Kolloquium ergibt je Praxisprojekt weitere 2~CP. (\textit{Bitte beschränken Sie sich bei Ihrem Vortrag auf 20 Minuten.} Dir Vorträge werden oft in kleinere Blöcken von Vorträgen abgehalten für die jeweils 30 Minuten eingeplant werden.)
\end{itemize}


\section{Lehrprojekt (Master of Information Engineering)}

\begin{itemize}
    \item Es werden 12 CP vergeben. Das entspricht einem Arbeitsaufwand von ca. 300 Stunden = \textbf{7,5 Wochen} (netto).
    Es findet kein Kolloquium statt.
    \item Bzgl. Themenvergabe, Betreuung, Terminen und abzugebende Dokumente gelten dieselben Richtlinien wie für Praxisprojekte (siehe Abschnitt~\ref{sec:praxisprojekt}).
\end{itemize}


\section{Bachelor-Thesis}

\begin{itemize}
    \item Die Bachelor-Thesis (Bachelor-Abschlussarbeit) wird in der Regel im 6. Praxisquartal angefertigt. Es werden 12 CP vergeben.
    
    \item Im Kolloquium verteidigen Sie die Inhalte Ihrer Thesis in einem Abschlussvortrag oder einem Prüfungsgespräch mit den beiden Gutachtern (s. auch Abschitt~\ref{sec:bewertung-kolloquium}).
    Der Erstprüfer legt die Modalität des Kolloquiums fest. Klären Sie die Modalität frühzeitig mit Ihrem Erstprüfer. Ein erfolgreich erbrachtes Kolloquium ergibt weitere 4 CP.
    
    \item Die Bearbeitungszeit für die Abschlussarbeit beträgt \textbf{10 Wochen}. Die Termine entnehmen Sie bitte dem Zeitplan des jeweiligen Quartals. Verlängerungen bis maximal drei Wochen können in besonderen Fällen beantragt werden. Ein entsprechender Antrag muss 3 Wochen vor Ende der regulären Bearbeitungszeit in Absprache mit dem Erstgutachter gestellt werden. Den genauen spätesten Zeitpunkt hierfür finden Sie im Intranet der FHDW Hannover in den Informationen zu Bachelor- und Master-Thesis.
    Ausfälle durch Krankheit während der Bearbeitungszeit melden Sie unverzüglich mit einem ärztlichen Attest der Verwaltung.
    
    \item \textbf{Tipp}: Zur Absprache der Themenstellung verfassen Sie vor dem Beginn der Bachelor-Thesis eine \textbf{Kurz-Exposé} des geplanten Inhalts inklusive eines groben Projektplans. Besprechen Sie dieses Kurz-Exposé mit Ihrem Betreuer und ggf. Ihrem Ansprechpartner im Partnerunternehmen. Sie können folgende \LaTeX{}-Vorlage dafür nutzen:
    
    \begin{center}
    \url{https://www.overleaf.com/read/hmpmfjprfzqf}.
    \end{center}

\end{itemize}


\section{Master-Thesis}

\begin{itemize}
    \item Die Master-Thesis (Master-Abschlussarbeit) wird im Anschluss an das 3. Mastersemester angefertigt. Es werden 16 CP vergeben. Ein erfolgreich erbrachtes Kolloquium ergibt weitere 4 CP.
    \item Die Bearbeitungszeit für die Abschlussarbeit beträgt \textbf{16 Wochen}.
    \item Bzgl. Terminen, Verlängerungen, abzugebende Dokumente, Art des Kolloquiums gelten die selben Regeln wie bei der Bachelor-Thesis.
    
    \item \textbf{Tipp}: Insbesondere bei einer Master-Thesis raten wir Ihnen zur schriftlichen Absprache der Themenstellung mit Ihrem Erstprüfer und ggf. dem Praxisbetreuer mittels des \textbf{Kurz-Exposés} (siehe oben).
\end{itemize}


\section{Transferprojekt}

\begin{itemize}
    \item Das Transferprojekt ist Teil des Aufbaustudiums als Vorbereitung für einen Masterstudiengang. Es werden 10 CP vergeben.
    \item Weitere Details zum Ablauf besprechen Sie mit Ihrem Prüfer.
\end{itemize}