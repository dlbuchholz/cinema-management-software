\chapter{Ein Anhang}
\label{chap:ein-anhang}


Der Anhang stellt eine Ergänzung zur eigentlichen Arbeit dar. Denkbare Einsatzmöglichkeiten sind:

\begin{itemize}
    \item Dokumentation von Materialien, die dem Leser üblicherweise nicht zugänglich sind (Herstellung der Zitierfähigkeit): Dieser Punkt kommt bspw. zum Tragen, wenn in der Arbeit unternehmensinterne Unterlagen eingearbeitet werden (z.B. Organigramme, Auszüge aus Organisationshandbüchern, Texte von Betriebsvereinbarungen, Verkaufsunterlagen, Produktbeschreibungen).
    \item Detaillierte Beschreibungen von hergestellten Artefakten, z.B. ausführliche Anwendungsfallbeschreibungen, wenn im Hauptteil Anwendungsfälle schon grob beschrieben sind und dies für das primäre Verständnis ausreicht, oder ausführliche Klassendiagramme, die übersichtsartige Diagramme im Hauptteil ergänzen. Hier können auch Code-Listings, von Algorithmen oder Skripten, die Sie entwickelt haben.
    \item Ergebnisse von Gesprächen mit Experten in Form von Gesprächsnotizen oder  Interviewprotokolle (z. B. in Frage-Antwort-Form).
\end{itemize}


Ein Anhang kann eine sinnvolle Ergänzung der eigentlichen Arbeit sein. Material sollte jedoch nicht beliebig angehängt werden, sondern durch geeignete Beschreibungen mit dem Inhalt der Arbeit in Verbindung gebracht werden. Es gilt auch hier: Beschränken Sie sich auf Wichtiges. Im Regelfall sollte der Anhang nicht 50\% des Seitenumfangs des Inhaltsteils überschreiten. Im Zweifel kann eine Arbeiten auch ohne Anhang auskommen.

\textbf{Der Anhang ist insbesondere kein Ort für Abbildungen, die aus dem Hauptteil ausgelagert werden} -- insbesondere nicht zu dem Zweck, dadurch den vorgegebenen Seitenumfang des hauptinhaltlichen Teils einzuhalten. Es erschwert unnötig die Lesbarkeit der Arbeit, wenn zwischen Hauptteil und Anhang hin- und hergeblättert werden muss. \textbf{Die Arbeit muss auch ohne Anhang vollständig und verständlich sein!}
